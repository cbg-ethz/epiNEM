%\VignetteEngine{knitr::knitr}
%% LyX 2.1.4 created this file.  For more info, see http://www.lyx.org/.
%% Do not edit unless you really know what you are doing.
\documentclass{article}
\usepackage[sc]{mathpazo}
\usepackage[T1]{fontenc}
\usepackage{geometry}
\geometry{verbose,tmargin=2.5cm,bmargin=2.5cm,lmargin=2.5cm,rmargin=2.5cm}
\setcounter{secnumdepth}{2}
\setcounter{tocdepth}{2}
\usepackage{url}
\usepackage[utf8]{inputenc}
\usepackage[unicode=true,pdfusetitle,
 bookmarks=true,bookmarksnumbered=true,bookmarksopen=true,bookmarksopenlevel=2,
 breaklinks=false,pdfborder={0 0 1},backref=false,colorlinks=false]
 {hyperref}
\hypersetup{
 pdfstartview={XYZ null null 1}}
\begin{document}



\title{Epistatic Nested Effects Models\\
	Inferring mixed epistatis from indirect measurements of knockout screens.}


\author{Madeline, Diekmann \& Martin Pirkl}

\maketitle
This package is an extension of the classic Nested Effects Models provided in package \emph{nem}. Nested Effects Models is a pathway reconstruction method, which takes into account effects of downstream genes. Those effects are observed for every knockout of a pathway gene, and the nested structure of observed effects can then be used to reconstruct the pathway structure.
However, classic Nested Effects Models do not account for double knockouts. In this package \emph{epiNEM}, one additional layer of complexity is added. For every two genes, acting on one gene together, the relationship is evaluated and added to the model as a logic gate. Genetic relationships are represented by the logics OR (no relationship), AND (functional overlap), NOT (masking or inhibiting) and XOR (mutual prevention from acting on gene C).

\section*{Loading epiNEM}



















